% fov_rt_et_BACKGROUND.tex
\section{Background}
\subsection{Ray tracing}
Ray tracing is an image synthesis technique in which rays are used to determine scene- and geometry visibility.
In computer graphics, rays often originate from the observer view-point and framebuffer texel grid.
Each ray subsequently identifies the closest scene element it may intersect, from which lighting, shadows, and other effects may be computed.
While ray tracing only recently having become sufficiently fast for real-time computer graphics such as video games, real-time ray tracing has been around for a long time; some crediting the 1987 BRL-CAD system as the first real-time ray tracer\cite{stay87}.

\subsection{Eye tracking}
Eye tracking is the process of sampling an observers gaze, often in relation to a surface such as a computer screen.
For the purposes of eye tracking, one may use an eye tracker, which is a device used to establish and measure positions of an observers eyes\cite{duchowski07}; such as \textit{fixations} and intermediate \textit{saccades}\cite{rayner98}.
The most common eye tracking device type used today is possibly the optical eye tracker; which uses a sequence of video images to measure movement of the eyes\cite{duchowski07}.
Recently, eye tracking technologies - although not a new subject to researchers - have become more accessible to the general public, mayhaps most notably with the recent Tobii EyeX Controller devkit.

\subsection{Limitations in human vision}
\ldots

\subsection{Foveated rendering}
\ldots
