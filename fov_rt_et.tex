% fov_rt_et.tex
\input{EGauthorGuidelines-cgf-sub.tex}

\usepackage{gensymb} % \degree\ command.
\usepackage{algorithm} % Algorithm environment for pseudocode.
\usepackage{algpseudocode} % Pseudocode package.
\usepackage{graphicx} % PNG figures.
\usepackage{hyperref} % Urls.
\usepackage{float} % Floaty fine figures.
\usepackage{stfloats} % To allow placement of float* environment at bottom of page.
\usepackage{pgf} % Used to round numbers to given number of significant figures.

\newcommand*{\no}[1]{%
    \pgfmathprintnumber[
        fixed,
        precision=3,
        fixed zerofill=true,
        ]{#1}}%

\restylefloat*{figure}
\restylefloat*{table}

\let\OLDthebibliography\thebibliography
\renewcommand\thebibliography[1]{
  \OLDthebibliography{#1}
  \setlength{\parskip}{0pt}
  \setlength{\itemsep}{0pt plus 0.3ex}
}

\title[Foveated Real-time Ray Tracing]{Foveated Real-time Ray Tracing}

\author[E. Nilsson]
       {E. Nilsson$^{1}$\\
        $^1$Blekinge Institute of Technology, Karlskrona, Sweden
       }

\begin{document}

\maketitle

% fov_rt_et_ABSTRACT.tex
\abstract{\ldots}

% fov_rt_et_INTRODUCTION.tex
\section{Introduction}
Ray tracing is an image synthesis technique in which rays are used to determine scene- and geometry visibility.
In computer graphics, rays typically originate from the observer view-point, based on the framebuffer texel grid, to determine framebuffer colors.
Each ray subsequently identifies the closest scene element it may intersect, from which lighting, shadows, and other effects may be computed.

Ray tracing has long since been used for computer graphics, both in offline and real-time scenarios, some real-time use-cases stretching as far back as a 1987 CAD-application~\cite{stay87}.
Recently, the fast progression of modern processors have accomodated for more extensive use of ray tracing technologies in real-time, as demonstrated by Intel Corporation with Quake Wars: Ray Traced in 2009~\cite{pohl09}, an implementation of the Id Software video game Enemy Territory: Quake Wars, again popularizing the idea of ray tracing renderers for consumer video games.

On the contrary, modern real-time graphics commonly employ the rasterization rendering technique.
Rasterization technologies, which are often significantly faster than ray tracing algorithms to map scene geometry to a computer screen, require complicated methods to approximate effects such as shadows or reflections; effects that may be described as trivial in terms of ray tracing.
Yet, rasterization techniques are often preferred to ray tracing renderers due to their superior speed and 'good-enough' results.

However, due to the continuing increase in display resolution, in particular the emergence of 4K displays, computer graphics are struggling to keep up with the increasing number of pixels to process.
To address this problem, Guenter et al. demonstrates that the computational complexity of high resolution imagery for real-time rasterization may be drastically reduced using foveation in coagency with modern eye tracking devices~\cite{guenter12}.

In 2012, Garc\'ia et al. presented a ray tracing model using DirectCompute~\cite{garcia12}; thus utilizing modern throughput-oriented many-core devices to accomplish ray tracing in real-time.
During implementation of a similar algorithm for a university project, we discovered that our video card was not keeping up to speed with higher resolutions.
As such, we decided to accelerate the ray tracing algorithm by employing foveation to reduce GPU workload, and in turn investigate whether or not the benefits foveation has for rasterization, established by Guenter et al., extends to ray tracing and to determine the impact such acceleration may have on rendering performance.

%% Some of the performance gains rasterization methodologies often have over ray tracing is that of efficient data coherence, the lack of which may induce high computational complexity.
%% While the rasterization process may cache and share computed results, ray tracing techniques commonly eveluate each ray individually; especially considering each ray may collide with different geometry.
%% Furthermore, complex geometry stresses the importance of acceleration structures for real-time ray tracing.
%% While acceleration structures, such as BVH trees, may offer great performance improvements, high-resolution render targets, such as emergent 4k resolutions, may come at high initial costs.

% fov_rt_et_BACKGROUND.tex
\section{Background}
\subsection{Ray tracing}
Ray tracing is an image synthesis technique in which rays are used to determine scene- and geometry visibility.
In computer graphics, rays often originate from the observer view-point based on the framebuffer texel grid in order to determine framebuffer colors.
Each ray subsequently identifies the closest scene element it may intersect, from which lighting, shadows, and other effects may be computed.

On the contrary, modern real-time graphics commonly use the rendering technique of rasterization.
Rasterization technologies, which are often significantly faster than ray tracing methodologies to map scene geometry to a computer screen, often require complicated methods to approximate (often roughly) effects such as shadows or reflections; effects that might be described as trivial to achieve in terms of ray tracing.
Yet, rasterization techniques are often preferred to ray tracing renderers due to their superior speed and 'good-enough' results.

Some of the performance gains rasterization methodologies often have over ray tracing is that of efficient data coherence, the lack of which may induce high computational complexity.
While the rasterization process may cache and share computed results, ray tracing techniques commonly eveluate each ray individually; especially considering each ray may collide with different geometry.
Furthermore, complex geometry stresses the importance of acceleration structures for real-time ray tracing.
While acceleration structures, such as BVH trees, may offer great performance improvements, high-resolution render targets, such as emergent 4k resolutions, may come at high initial costs.

\subsection{Eye tracking}
Eye tracking is the process of sampling an observers gaze, often in relation to a surface such as a computer screen.
For the purposes of eye tracking, one may use an eye tracker, which is a device used to establish and measure positions of an observer's eyes\cite{duchowski07}; such as \textit{fixations} and intermediate \textit{saccades}\cite{rayner98}.
The most common eye tracking device type used today is possibly the optical eye tracker; which uses a sequence of video images to measure movement of the eyes\cite{duchowski07}.
Recently, eye tracking technologies - although not a new subject to researchers - have become more accessible to the general public, mayhaps most notably with the recent Tobii EyeX Controller devkit.

\subsection{Human vision}
Human vision may be thought of as being subdivided in three distinct zones; the foveal, parafoveal, and peripheral vision\cite{rayner98}.
The fovea, which makes out roughly $2$\degree\ of human eye-sight, is the area in which the human eye features the outmost acuity due to vast concentrations of visual receptors.
As such, we position the fovea on the stimulus we wish to observe.
Along the outer reaches of the fovea, and extending $5$\degree\ from the center of the fovea, lie the parafovea which, while still quite acute, is considerably less keen than the fovea.
The remaining (majority) of the human vision we may classify as peripheral vision which, while not acute as fovea or parafovea, constitutes the remaining visual field of roughtly $135$\degree\ vertical- and $160$\degree\ horizontal vision\cite{guenter12}.
We may refer to the decrease of the receptive qualities of human eye-sight as foveation.

\subsection{Foveated rendering}
In computer graphics, aforementioned eye-sight limitations are often ignored; real-time graphics applications, such as video games, rendering complex scenes at full-resolution independently of where the observer is looking.
One might argue that vast amounts of processing power is spent on rendering high resolution imagery for human peripheral vision which may not necessarily appreciate high image quality.
After all, a $5$\degree\ parafoveal area is likely a small component of most display systems.

Hence, in real-time graphics - where time is of the essence - it may be benificial to render only those parts of a framebuffer which the eye may appreciate at high quality; reducing rendering complexity for a potentially large area of the display.
Utilizing foveation in image processing is not a new idea\cite{levoy90}, yet one that seems to have escaped developers of real-time graphics, possibly due to the lack of low-latency eye tracking devices in consumer markets.

% fov_rt_et_CONTRIBUTION.tex

\section{Contribution}

\begin{figure*}
\parbox{.5\linewidth}{%
\centering%
\resizebox{1.0\linewidth}{!}{\input{gnu_non-foveated.tex}}
\caption{Non-foveated ray tracing.}
\label{fig:histogram_non-foveated}}
\hfill%
\parbox{.5\linewidth}{%
\centering%
\resizebox{1.0\linewidth}{!}{\input{gnu_foveated.tex}}
\caption{Foveated ray tracing.}
\label{fig:histogram_foveated}}
\end{figure*}

The ray tracing renderer devised for the purpose of this study adheres to the model presented by Garc{\'i}a et al.: devised in DirectCompute using C$++$.
The algorithm renders the scene with the help of three distinct steps that compiles a number of rays equal to that of the application window resolution, performs intersections tests on each ray with scene geometry to establish the closest intersected geometry, and finally draws the scene to the framebuffer using scene lights- and geometry in order to establish what areas of the scene are in shadow.
The ray tracing algorithm is presented as pseudocode in algorithm \ref{algrt}.
Note that some elements of the algorithm are not featured in-pseudo-code, such as the comparison of closest intersected scene geometry, in order to keep the code short and concise.

\begin{algorithm}
\begin{algorithmic}[1]
\Procedure{raytrace}{$rays, reflCnt$}
\caption{Ray tracing algorithm}\label{algrt}
\State $rays\gets$\Call{GenRays}{screen, frustum}
\While{$reflCnt>0$}
\ForAll{$rays$}\Comment{for each pixel}
\If {\Call{intersects}{ray, objs}}
    \State $obj\gets objs$
\EndIf
%\Require $obj$ % If you're to be entirely precise.
\ForAll{$lights$}
\State $color\gets$\Call{Shadow}{ray, obj}
\State $color\gets$\Call{Lighting}{obj}
\EndFor
\State $backbuffer\gets color$
\EndFor
\State $reflCnt\gets reflCnt - 1$
\EndWhile
\EndProcedure
\end{algorithmic}
\end{algorithm}

In order to optimize this algorithm to make it run sufficiently fast on the platform at hand we made use of a Tobii EyeX Dev Kit to establish where on the screen the observer is focusing his or her gaze.
Using this information, we may render only parts of the application window - where the user has his or her gaze fixation - at full resolution; rendering those areas of the window in the user's peripheral vision at a lower resolution.
Since, as described in algorithm \ref{algrt}, the total number of computed rays is based off the texel grid of the framebuffer this may drastically reduce the number of rays required to render a scene.

\subsection{Hardware}
For the purpose of this experiment, we make use of the Tobii EyeX Devkit Controller, which is a consumer-level corneal-reflection eye tracking device, which may the position the gaze of an oberserver on a computer screen.
The EyeX controller, while still a developer's prototype variant, is a fairly competent device; the consumer version expected to be released later this year.

The experiment is performed on a Windows~8.1 system with the following specifications:
\begin{itemize}
\setlength\itemsep{0em}
\item Intel Q9550 Quad Core 2.83GHz
\item ATI Radeon HD 5800
\end{itemize}

For the screen, we use a $23''$ $510$mm~$\times$~$287$mm Samsung~Syncmaster~$2343$ monitor, beneath which the eye tracking device is placed for the duration of the experiment.

\subsection{Software}
For retrieval of gaze positional data and communication with the eye tracking device, we utilize the Tobii C/C$++$~SDK.

In order to achieve varying levels of quality - or resolution - in the field-of-vision, the rendering process is subdivided into a number of 'FOV's (short for fovea, parafovea, etc.); each it's own render target.
Using an arbitrary of number these FOVs, we may vary the quality of the rendered scene accross the user's field of vision.
For the purpose of this study, although FOVs for both foveal-, parafoveal-, and peripheral vision have been prototyped, we limit the variance in quality to that of parafoveal and peripheral vision.
Thus, we render the scene at two different resolutions during the experiment; corresponding to parafoveal- and peripheral vision.

This is due to the EyeX controller device.
Being a consumer-level developer's kit, the device tracking may bring about a jittery render target when the designated area is as small as $2$\degree\ (human foveal vision) of the observer's view.
As this is considerably less noticable with a larger area (giving the eye tracking device more time to garner gaze positional data when the user moves his or her eyes); rendering the entirety of the parafoveal vision (5\degree ) at high quality was deemed more appropriate.
After all, the purposes of foveated rendering is to reduce computational complexity whilst maintaining percieved scene quality.

For our setup, the observer is positioned 700mm from the screen (which is the distance relayed appropriate by the device).
The application window resolution is that of $1152\times 1152$, a resolution chosen due to it being the highest our system setup could display whilst keeping the resolution square.
A $5$\degree\ parafovea thus bring about an area of roughly $31$mm$\times $$31$mm, which represents, for the utilized screen, roughly $123\times 123$~pixels; a small percentage of the application window resolution.
Note, however, that this component would be considerably smaller should the application run at even higher resolutions.
In fact, presuming the same system setup and distance to the screen, the pixel resolution within the observer's foveal vision should remain constant.

For the peripheral vision, or the low-quality render target, we chose to render the scene at only a fourth of the full-resolution.
Accordingly, our rendering algorithm is performed in an additional two steps; wherein the scene is rendered at full-resolution in a grid of only $123\times 123$ pixels, while the rest of the framebuffer is rendered at a fourth of full-resolution and subsequently upscaled to fit the application window.
The high-resolution parafoveal framebuffer is then copied onto the screen framebuffer at the location where the observer is currently focusing his or her gaze.

\begin{figure}[p]
  \centering
  \includegraphics[width=1.0\linewidth]{img/fov_rt_et.png}
  \caption{An $800\times 800$ foveated scene. The observer's gaze is positioned on the cube. Note that the surrounding scenery, including the cube's reflection in the plane, is rendered at a lower resolution. This particular scene, on the contrary to the experiment, renders the peripheral areas in an eighth of the full-resolution in order to make the foveated effect more prominent and noticable for the sake of visualization.}
  \label{fig:fov}
\end{figure}

% fov_rt_et_CONCLUSION.tex

\section{Conclusion}
The scene used for the purpose of this experiment is a mesh consisting of $12$ vertices that are indexed to create $14$ triangles, which are scaled, rotated, and translated into the scene.
This mesh, although positioned in a different manner in relation to the camera, is presented in figure~\ref{fig:fov}.
Furthermore, the scene is lit using three pointlights; each of which may cast shadows.

In order to render the scene, which rotates every given frame, the renderer computes three reflections thus repeating the intersection- and lighting stages of the algorithm three times per pixel.
During the experiment, the higher-resolution parafoveal render target was locked to be positioned in the center the screen.
As such, the eye tracking had no influence on the results.

We collected elapsed times for each stage during $1000$ frames.
If a stage was computed several times (computing reflections with intersection- and lighting stages) the elapsed times were concatenated.
The mean values of these measurements are presented in milliseconds in table~\ref{tab:res}.

\begin{table}[h]
\begin{tabular}{lll}
  & Non-foveated & Foveated \\
  Generate Rays & 1.36\phantom{0} ms & 0.09 ms \\
  Intersection & 12.32 ms & 1.05 ms \\
  Lighting & 18.01 ms & 1.72 ms
\end{tabular}
\caption{Mean elapsed time in milliseconds for the ray tracing shader stages for non-foveated and foveated samples. Note that the latter two shader stages are performed multiple times (one per reflection); the elapsed times of which are concatenated for these samples.}
\label{tab:res}
\end{table}

From the measurements presented in table~\ref{tab:res}, we may establish that foveation reduced execution time of over $90\%$ in all three ray tracing stages.
This reduction was significant enough to run the ray tracer in high-definition resolutions, and thus made it possible for us increase the application resolution to above that of $800\times 800$ which we were previously limited to.

Furthermore, the foveated rendering is mostly transparent and not too noticable.
Although gaze position latency and aliasing issues in the low resolution render target (in the peripheral vision) is noticable, it does not particularily disrupt the observer's perception of the scene.
That being said, foveation accomodating for considerably better performance is a good-enough compromise.

\subsection{Future work}
Currently, the foveation methodology used causes somewhat severe overdraw, which could worsen should additional FOVs be added.
While the overdraw, with only peripheral and parafoveal render targets, causes only those rays positioned behind the high resolution FOV (spanning a fourth of the high-resolution number of rays) to be computed without having any impact on the scene, the number of redundant rays cast may be significantly increased with the addition of more FOVs, especially as more highly detailed FOVs would suffer overdraw.
Note, however, that the current implementation only causes overdraw of a portion of the low-resolution peripheral FOV.

Furthermore, and an issue more increasingly a problem with quality reduction of peripheral and parafoveal views, aliasing issues outside of the foveal area may cause jitter in the peripheral vision of the observer.
While advanced anti-aliasing techniques may be considered to lessen this issue, blurring might also be a viable candidate to reduce such jitter; especially along the borders of FOVs where varying levels of quality is made particularily apperent.

Additionally, while we may expect consumer-level eye tracking devices to improve in terms of gaze positional latency and accuracy, these attributes are vital to real-time rendering foveation in order to maintain the illusion of consistent quality in a scene.
The latency in garnering positional data for a gaze-contingent application, such as the one presented in this document, may cause the application to fall behind.
In terms of real-time rendering, this causes the high-quality render target that the user ought be seeing, to lag behind and thus causing the user to look upon a lower quality render of the scene he or she expects.
Most of the time, this is only an issue if the eye tracker temporarily cannot locate an observer's eyes, or in combination with accuracy issues.

Another issue presenting itself in terms of eye tracking hardware is that of gaze positional accuracy, which may cause the high quality render to be offset erronously in relation to an observer's gaze.
In particular, this was discovered to be an issue for especially small FOVs - such as an high-quality foveal render target (making out only $2$\degree\ of the viewer's vision).

However, these hardware issues are expected to improve in the near-future; hopefully as eye tracking devices become more common in consumer-markets.


\bibliography{fov_rt_et}{}
\bibliographystyle{eg-alpha}

\end{document}
